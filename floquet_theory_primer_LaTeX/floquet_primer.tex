\documentclass{article}
\usepackage[utf8]{inputenc}
\usepackage{textcomp}
\usepackage{amsmath}
\usepackage{amssymb} 
\usepackage[margin=1in]{geometry}
\usepackage{mathtools}
\usepackage{graphicx}
\usepackage{epstopdf}
\usepackage{pdfpages}
\graphicspath{ {../figures/} }
% \usepackage{caption}
% \usepackage{multirow}
% \usepackage{mwe}
% \usepackage{subfig}
% \usepackage{caption}
% \usepackage{float}
% \usepackage{cite}

\title{Floquet Theory Basics}
\author{Bryan Kaiser}
%\date{October 2016}

\begin{document}

\maketitle

Consider the non-autonomous system, 
\begin{equation}
 \frac{\mathrm{d}\mathbf{x}}{\mathrm{dt}}=\mathbf{A}(t)\mathbf{x}(t)
\end{equation}
where the operator $\mathbf{A}(t)$ is periodic:
\begin{equation}
 \mathbf{A}(t+T)=\mathbf{A}(t)
\end{equation}
Note that while $\mathbf{A}$ may be periodic, $\mathbf{x}$ need not be.
If
\begin{equation}
 \mathbf{x}=\begin{bmatrix*}
  u \\
  v
 \end{bmatrix*}
\end{equation}
then the general solution is a superposition of two solutions $\mathbf{x}_1$ and $\mathbf{x}_2$:
\begin{equation}
 \mathbf{x}(t)=c_1\mathbf{x}_1+c_2\mathbf{x}_2
 =c_1\begin{bmatrix*}
  u_1 \\
  v_1
 \end{bmatrix*}+c_2\begin{bmatrix*}
  u_2 \\
  v_2
 \end{bmatrix*}
\end{equation}
and $c_1$ and $c_2$ are arbitrary, time-independent constants.
The fundamental solution matrix is a matrix in which the columns are the solutions
\begin{equation}
 \boldsymbol{\Phi}(t)=\begin{bmatrix*}
 u_1 & u_2 \\
 v_1 & v_2
 \end{bmatrix*}
\end{equation}
The fundamental matrix $\Phi$ to the original system is \textit{not} unique, as there are many ways to 
choose independent solutions and arbitrary constants. 
%Since the solutions are linearly independent, we call them 
%a fundamental set of solutions and $\Phi$ is a fundamental 
%solution matrix. 
Therefore, we can write the general solution as:
\begin{equation}
 \mathbf{x}(t)=\boldsymbol{\Phi}(t)\mathbf{c}=\begin{bmatrix*}
 u_1 & u_2 \\
 v_1 & v_2
 \end{bmatrix*}
  \begin{bmatrix*}
  c_1 \\
  c_2
 \end{bmatrix*}
 \label{eq:gen_soln}
\end{equation}
Since the arbitrary constants $\mathbf{c}$ are independent of time, we 
can define them using (usually known) the initial conditions:
%At some initial time:
\begin{flalign}
 \mathbf{x}(0)
 %=
 % \mathbf{x}_0 \nonumber \\
  &=\boldsymbol{\Phi}(0)\mathbf{c} %\nonumber \\
  %&=\boldsymbol{\Phi}_0\mathbf{c}
 \label{eq:x_IC}
\end{flalign}
Since the determinant of the initial fundamental solution matrix, $\text{det}[\Phi_0]$, 
is the value at $t_0$ of the Wronskian of the 
independent solutions $\mathbf{x}_1$ and
$\mathbf{x}_2$, it is non-zero because the two solutions are linearly independent.
Therefore a matrix inverse exists and the constants $\mathbf{c}$ can be obtained:
\begin{equation}
 \mathbf{c}=\boldsymbol{\Phi}(0)^{-1}\mathbf{x}(0)
 \label{eq:obtain_c}
\end{equation}
Equation \ref{eq:obtain_c} can be substituted into Equation \ref{eq:gen_soln}
to obtain the forward propogator, which maps $\mathbf{x}_0$ onto $\mathbf{x}(t)$:
\begin{flalign}
 \mathbf{x}(t)&=\boldsymbol{\Phi}(t)\mathbf{c} \nonumber \\
              &=\boldsymbol{\Phi}(t)\boldsymbol{\Phi}(0)^{-1}\mathbf{x}(0)
 \label{eq:forward_propogator}
\end{flalign}
% to the first equation to form columns of .
% To define the fundamental matrix, we need to define it such that it's determinant 
% must be non-zero and it
Note that the fundamental solution 
matrix must also satisfy the equation:
\begin{equation}
 \frac{\mathrm{d}\boldsymbol{\Phi}}{\mathrm{dt}}=\mathbf{A}(t)\boldsymbol{\Phi}(t)
 \label{eq:matrix_equation}
\end{equation}
Since $\boldsymbol{\Phi}(t)$ is a fundamental matrix, then it can be shown that 
%so must be 
$\mathbf{Y}(t)=\boldsymbol{\Phi}(t)\mathbf{C}$ is also a fundamental matrix, so long as
$\mathbf{C}$ is a non-singular constant matrix. We can then 
choose $\mathbf{Y}(t)=\boldsymbol{\Phi}(t+T)$ to obtain:
\begin{equation}
 \boldsymbol{\Phi}(t+T)=\boldsymbol{\Phi}(t)\mathbf{C}
\end{equation}
% This equation is full set of linearly independent solutions 
% depending on the set of linearly independent initial conditions.
% If we integrate over one period:
%If $\Phi_0=\mathbf{I}$ then $\Phi(t)$ is the prinicipal fundamental matrix
%(every floquet 1). 
%where $\mathbf{C}$ is independent of time and non-singular. 
Now, \textit{if 
$\boldsymbol{\Phi}_0=\mathbf{I}$ then $\boldsymbol{\Phi}(T)$ 
is the prinicipal fundamental solution matrix}, denoted by $^*$:
\begin{flalign} 
 \mathbf{C}&=\boldsymbol{\Phi}^*(0)^{-1}\boldsymbol{\Phi}^*(T) \nonumber \\
           &=\mathbf{I}^{-1}\boldsymbol{\Phi}^*(T) \nonumber \\
           &=\boldsymbol{\Phi}^*(T) 
\end{flalign}
$\boldsymbol{\Phi}_0=\mathbf{I}$ can be imposed without a loss of generality because of 
there is no unique choice for $\boldsymbol{\Phi}$ for any given system.
Therefore the initial condition on $\mathbf{x}$ (Equation \ref{eq:x_IC}) is:
\begin{equation}
 \mathbf{x}_0=\mathbf{c}
\end{equation}
and the forward propogator equation (Equation \ref{eq:forward_propogator}) is:
\begin{flalign}
 \mathbf{x}(t)&=\boldsymbol{\Phi}^*(t)\mathbf{x}_0 %\nonumber \\
 %&=\boldsymbol{\Phi}^*(t)\mathbf{c}
\end{flalign}
and therefore
\begin{flalign}
 \mathbf{x}(T)&=\boldsymbol{\Phi}^*(T)\mathbf{x}_0 %\nonumber \\
 %&=\mathbf{C}\mathbf{x}_0 %\nonumber \\
 \label{eq:Cmapping}
\end{flalign}
Eigenanalysis of $\boldsymbol{\Phi}^*(T)$ selects the eigenvectors of $\boldsymbol{\Phi}^*(T)$ as the initial conditions for normal 
modes that are mapped from the initial conditions to the solutions $\mathbf{x}(T)$ by the eigenvalues 
of $\boldsymbol{\Phi}^*(T)$. 
The eigensystem:
% \begin{equation*}
%  \Phi(0)\mathbf{v}_k(0)=\mathbf{I}\mathbf{v}_k(0)=\mu_k\mathbf{I}\mathbf{v}_k(0)
% \end{equation*}
\begin{flalign}
\mathbf{x}(T)_k&= 
 \boldsymbol{\Phi}^*(T)\mathbf{x}_{0,k} \nonumber \\ &=\boldsymbol{\Phi}^*(T)\mathbf{v}_k \nonumber \\ &=\mu_k\mathbf{I}\mathbf{v}_k
 \nonumber \\ &=\mu_k\mathbf{x}_{0,k}
 %\mathbf{C}\mathbf{v}_k=
\end{flalign}
where subscript $k$ denotes each eigenvalue and eigenvector pair. This mapping
is the core of Floquet analysis. $\mu_k$ are the Floquet multipliers and $\mathbf{x}_k$ are 
the Floquet modes.
% can be generalized for all $t$:
% \begin{equation*}
%   \mathbf{x}(t)_k=\Phi(t)\mathbf{v}_k
% \end{equation*}
%  recall
% \begin{equation}
%  \Phi(t+T)=\Phi(t)\mathbf{C}
%  \label{eq:prinicipal_fundamental_matrix}
% \end{equation}
% Then we can say in general %(again taking $t_0=0$)
% \begin{equation*}
%   \mathbf{x}(t)_k=\Phi(t)\mathbf{v}_k
% \end{equation*}
% and from before:
% \begin{equation*}
%  \Phi(t+T)=\Phi(t)\mathbf{C}
% \end{equation*}
% Floquet modes:
% \begin{flalign*}
%  \mathbf{x}(t+T)_k&=\Phi(t+T)\mathbf{v}_k \nonumber \\
%               &=\Phi(t)\mathbf{C}\mathbf{v}_k \nonumber \\
%    &=\Phi(t)\mu_k\mathbf{I}\mathbf{v}_k \nonumber \\
%    &=\mathbf{x}(t)_k\mu_k
% \end{flalign*}
% The eigenvalues of $\mathbf{C}$ are the Floquet multipliers. 
% So, you can take
% \begin{flalign*}
%  \mathbf{x}(T)_k&=\Phi(T)\mathbf{v}_k \nonumber \\
%               &=\mathbf{C}\mathbf{v}_k \nonumber \\
%    &=\mu_k\mathbf{I}\mathbf{v}_k \nonumber \\
%    &=\mu_kc\mathbf{x}(0)_k
% \end{flalign*}
% where
% \begin{equation*}
%  \mathbf{x}(0)_k=\mathbf{I}\mathbf{v}_k
% \end{equation*}
% and $\mathbf{v}_k$ is independent of time.

Integration of the principal fundmental matrix $\Phi(t)$ (for $\Phi_0=\mathbf{I}$) over one 
period
\begin{equation}
 \int_0^T\mathrm{d\boldsymbol{\Phi}^*}=\boldsymbol{\Phi}^*(T)-\boldsymbol{\Phi}^*(0)=\boldsymbol{\Phi}^*(T)-\mathbf{I}
 =\int_0^T\mathbf{A}(t)\cdot\boldsymbol{\Phi}^*(0)\hspace{0.5mm}\mathrm{dt}
 =\int_0^T\mathbf{A}(t)\cdot\mathbf{I}\hspace{0.5mm}\mathrm{dt}%=
%  \begin{bmatrix*}
%   -\alpha{T} & \frac{i}{\omega}(\mathrm{e}^{i\omega{T}}-1) \\
%   0 & -\beta{T}
%  \end{bmatrix*} %\quad \text{where} \quad \phi(t)=\omega{t}=2\pi{t}/T
\end{equation}
therefore
\begin{equation}
 \boldsymbol{\Phi}^*(T)=\mathbf{I}+\int_0^T\mathbf{A}(t)\hspace{1mm}\mathrm{dt}
 \label{eq:PhiT}
\end{equation}

\newpage
\section*{An example 2$\times$2 system}
Let the system of equations be:
\begin{equation}
 \frac{\partial{u}}{\partial{t}}=-\alpha u-v\mathrm{e}^{i\omega{t}},
 \qquad \frac{\partial{v}}{\partial{t}}=-\beta{} v
 \label{eq:example}
\end{equation}
% \begin{equation*}
%  \mathbf{x}(t)=
%  \begin{bmatrix*}
%   -\alpha & -\mathrm{e}^{i\omega{t}} \\
%   0 & -\beta
%  \end{bmatrix*} %\quad \text{where} \quad \phi(t)=\omega{t}=2\pi{t}/T
% \end{equation*}
therefore
\begin{equation}
 \mathbf{A}(t)=
 \begin{bmatrix*}
  -\alpha & -\mathrm{e}^{i\omega{t}} \\
  0 & -\beta
 \end{bmatrix*} %\quad \text{where} \quad \phi(t)=\omega{t}=2\pi{t}/T
\end{equation}
and the solutions are:
% \begin{equation*}
%   v(t)=v_0\mathrm{e}^{-\beta{t}}
% \end{equation*}
% and
% \begin{equation*}
%  u(t)=u_0\mathrm{e}^{-\alpha{t}}+\frac{v_0}{\beta-i\omega-\alpha}\mathrm{e}^{(i\omega-\beta){t}}
% \end{equation*}
\begin{equation}
 \mathbf{x}(t)=
 \begin{bmatrix*}
  u_0\mathrm{e}^{-\alpha{t}}+\frac{v_0}{\beta-i\omega-\alpha}\mathrm{e}^{(i\omega-\beta){t}} \\
  v_0\mathrm{e}^{-\beta{t}}
 \end{bmatrix*}
 \label{eq:analytical_soln}
\end{equation}
for the initial conditions
\begin{equation}
 \mathbf{x}(0)=
 \begin{bmatrix*}
  u_0+\frac{v_0}{\beta-i\omega-\alpha} \\
  v_0
 \end{bmatrix*}
\end{equation}
% Elementwise solutions to Equation \ref{eq:matrix_equation} for the principal solution matrix for 
% this particular case:
% % \begin{equation}
% %  \frac{\partial\Phi_{ij}}{\partial{t}}={A}_{ij}\Phi_{ij}
% % \end{equation}
% are
% \begin{flalign}
%  \Phi_{ij}^*(t)&=\mathrm{e}^{{A}_{ij}}\Phi_{ij}^*(0) \nonumber \\
%   &=\mathrm{e}^{{A}_{ij}}I_{ij}(0)
% \end{flalign}
% therefore 
The periodic term in $\mathbf{A}$ integrates to zero, therefore the 
principal fundamental solution matrix (denoted by $^*$) is
\begin{equation}
 \boldsymbol{\Phi}^*(T)
%  =\begin{bmatrix*}
%   1-\alpha{T} & \frac{i(\mathrm{e}^{i\omega{T}}-1)}{\omega} \\
%   0 & 1-\beta{T}
%  \end{bmatrix*}
 =\begin{bmatrix*}
  \mathrm{e}^{-\alpha{T}} & 0 \\
  0 & \mathrm{e}^{-\beta{T}}
 \end{bmatrix*}
\end{equation}
which has the eigenvalues (Floquet multipliers):
% \begin{equation*}
%  (1-\alpha{T}-\lambda)(1-\beta{T}-\lambda)=0
% \end{equation*}
\begin{equation}
 \mu_1=\mathrm{e}^{-\alpha{T}} \qquad \mu_2=\mathrm{e}^{-\beta{T}}
 \label{eq:multipliers}
\end{equation}
and the eigenvectors (initial conditions for the Floquet modes):
\begin{equation}
 \mathbf{v}_1=\begin{bmatrix*}
  1 \\
  0 
 \end{bmatrix*}
\qquad
 \mathbf{v}_2
%  =\begin{bmatrix*}
%   \frac{i(\mathrm{e}^{i\omega T}-1)}{\omega{T}(\alpha-\beta)} \\
%   1 
%  \end{bmatrix*}
 =\begin{bmatrix*}
  0 \\
  1 
 \end{bmatrix*}
 \label{eq:eigenvecs}
 \end{equation}
 The two independent solutions of $\mathbf{x}(T)$ can be use to form another fundamental solution matrix:
%  \begin{equation}
%  \mathbf{x}(T)=
%  \begin{bmatrix*}
%   u_0\mathrm{e}^{-\alpha{T}}+\frac{v_0}{\beta-i\omega-\alpha}\mathrm{e}^{-\beta{T}} \\
%   v_0\mathrm{e}^{-\beta{T}}
%  \end{bmatrix*}
% \end{equation}
%  are
%  \begin{equation}
%  \mathbf{x}(T)_1=
%  \begin{bmatrix*}
%   u_0\mathrm{e}^{-\alpha{T}} \\
%   0
%  \end{bmatrix*}
% \end{equation}
% \begin{equation}
%  \mathbf{x}(T)_2=
%  \begin{bmatrix*}
%   \frac{v_0}{\beta-i\omega-\alpha}\mathrm{e}^{-\beta{T}} \\
%   v_0\mathrm{e}^{-\beta{T}}
%  \end{bmatrix*}
% \end{equation}
% such that 
\begin{equation}
 \mathbf{x}(T)=\boldsymbol{\Phi}(T)\mathbf{c}=\begin{bmatrix*}
  \mathrm{e}^{-\alpha{T}} & \frac{\mathrm{e}^{-\beta T}}{\beta-i\omega-\alpha} \\
  0 & \mathrm{e}^{-\beta{T}}
 \end{bmatrix*}
 \begin{bmatrix*}
  u_0 \\
  v_0
 \end{bmatrix*}
 \label{eq:final_matrix}
\end{equation}
Note that this fundamental solution matrix is not the prinicipal fundamental solution matrix because 
it could not arise from the initial condition $\Phi_0=\mathbf{I}$. Both sets of initial conditions from 
the eigenvectors of the prinicipal fundamental solution matrix (Equations \ref{eq:eigenvecs}) can be used to map the 
the solutions from time $t=0$ to time $t=T$, as 
 in Equation \ref{eq:Cmapping}:
 \begin{equation}
  \mathbf{x}(T)_k=\boldsymbol{\Phi}^*(T)\mathbf{v}_k
 \end{equation}
Floquet mode $k=1$ %corresponds to selecting $u_0=0,v_0=1$:
 \begin{equation}
  %\mathbf{x}(T)_k&=\boldsymbol{\Phi}^*(T)\mathbf{v}_k \nonumber \\
  \mathbf{x}(T)_1=\boldsymbol{\Phi}^*(T)\mathbf{v}_1=\begin{bmatrix*}
  \mathrm{e}^{-\alpha{T}} & 0 \\
  0 & \mathrm{e}^{-\beta{T}}
 \end{bmatrix*}
 \begin{bmatrix*}
  1 \\
  0 
 \end{bmatrix*}
 =\mu_1\mathbf{v}_1
 \end{equation}
 corresponds to selecting:
 \begin{equation}
  u_0=0,\qquad v_0=1
 \end{equation}
 in Equation \ref{eq:analytical_soln}.% at time $t=T$.
Floquet mode $k=2$
 \begin{equation}
  %\mathbf{x}(T)_k&=\boldsymbol{\Phi}^*(T)\mathbf{v}_k \nonumber \\
  \mathbf{x}(T)_2=\boldsymbol{\Phi}^*(T)\mathbf{v}_2=\begin{bmatrix*}
  \mathrm{e}^{-\alpha{T}} & 0 \\
  0 & \mathrm{e}^{-\beta{T}}
 \end{bmatrix*}
 \begin{bmatrix*}
  0 \\
  1 
 \end{bmatrix*}
 =\mu_2\mathbf{v}_2
 \end{equation}
corresponds to selecting
\begin{equation}
 u_0=-\Big(\frac{\mathrm{e}^{-\beta{T}}}{\beta-i\omega-\alpha}+\mathrm{e}^{-\alpha{T}}\Big), \qquad v_0=1
\end{equation}
in Equation \ref{eq:analytical_soln}.% at time $t=T$.
The Floquet exponents are defined as 
\begin{equation*}
 \mu_k=\mathrm{e}^{\gamma_k{T}}
\end{equation*}
which indicate Floquet mode frequencies exist for imaginary $\gamma_k$ and 
that if the real part of $\gamma_k$ is greater than unity growth will occur.
%  \begin{equation*}
%   \begin{bmatrix*}
%   u_0\mathrm{e}^{-\alpha{T}}+\frac{v_0}{\beta-i\omega-\alpha}\mathrm{e}^{-\beta{T}} \\
%   v_0\mathrm{e}^{-\beta{T}}
%  \end{bmatrix*}
%  =\begin{bmatrix*}
%   \mathrm{e}^{-\alpha{T}} & 0 \\
%   0 & \mathrm{e}^{-\beta{T}}
%  \end{bmatrix*}
%  \begin{bmatrix*}
%   1 \\
%   0 
%  \end{bmatrix*}
%  \end{equation*}
%  \begin{equation*}
%   \begin{bmatrix*}
%   u_0\mathrm{e}^{-\alpha{T}}+\frac{v_0}{\beta-i\omega-\alpha}\mathrm{e}^{-\beta{T}} \\
%   v_0\mathrm{e}^{-\beta{T}}
%  \end{bmatrix*}
%  =\begin{bmatrix*}
%   \mathrm{e}^{-\alpha{T}} & 0 \\
%   0 & \mathrm{e}^{-\beta{T}}
%  \end{bmatrix*}
%  \begin{bmatrix*}
%   0 \\
%   1 
%  \end{bmatrix*}
%  \end{equation*}
 
\begin{figure}[h!]
\centering
\includegraphics[width=4.75in]{local_loglog_error.png}
\caption{$L_\infty$ norm of the local (i.e. one time step) Floquet multiplier percent error 
($\max[|\lambda_{k,\text{computed}}-\lambda_{k,\text{analytical}}|/|-\lambda_{k,\text{analytical}}|]$) for the system of equations 
in Equation \ref{eq:example} as a function of time step divided by period. 
The principal fundamental solution matrix $\boldsymbol{\Phi}^*(t)$
was advanced in time explicitly using a 4th order Runge Kutta method (for which the local truncation error is $\mathcal{O}(\Delta{t}^4)$) one time step $\Delta{t}$ 
and then the 
percent error was computed between the
eigenvalues 
(i.e. Floquet multipliers) of 
the computed matrix $\boldsymbol{\Phi}^*(t)$ and the exact analytical solutions (Equation \ref{eq:multipliers}). Machine precision is acheived at time step 
size corresponding to about $3\cdot10^4$ time steps per period.}
\end{figure}


\end{document}
