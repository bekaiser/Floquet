\documentclass{article}
\usepackage[utf8]{inputenc}
\usepackage{textcomp}
\usepackage{amsmath}
\usepackage{amssymb} 
\usepackage[margin=1in]{geometry}
\usepackage{mathtools}
\usepackage{graphicx}
\usepackage{epstopdf}
\usepackage{pdfpages}
\graphicspath{ {figures/} }
% \usepackage{caption}
% \usepackage{multirow}
% \usepackage{mwe}
% \usepackage{subfig}
% \usepackage{caption}
% \usepackage{float}
% \usepackage{cite}
\newcommand\myeq{\stackrel{\mathclap{\normalfont\mbox{\scriptsize{def}}}}{=}}

\title{Adaptive Runge-Kutta methods}
\author{Bryan Kaiser}
%\date{October 2016}

\begin{document}

\maketitle

Advance two exact solutions from $t$ to $t+\Delta{t}$. Advance one by taking one time step of magnitude 
$\Delta{t}$ (called $\mathbf{x}_1$) 
and 
the other by taking two time steps of $\Delta{t}$ (called $\mathbf{x}_2$), such that:
\begin{flalign*}
 \mathbf{x}(t+\Delta{t}) &= \mathbf{x}_1+\mathbf{c}(\Delta{t})^{p+1}+\mathcal{O}(\Delta{t}^{p+2}) \\
 \mathbf{x}(t+\Delta{t}) &= \mathbf{x}_2+\mathbf{c}2(\Delta{t}/2)^{p+1}+\mathcal{O}(\Delta{t}^{p+2})
\end{flalign*}
where the value $\mathbf{c}$ remains constant over the step and $\mathcal{O}(\Delta{t}^{p+2})$ represents 
the sixth and higher order terms. The difference between the two equations is an indicator 
of the truncation error:
\begin{flalign*}
 0&=\mathbf{x}_1+\mathbf{c}(\Delta{t})^{p+1}-\mathbf{x}_2-\mathbf{c}2(\Delta{t}/2)^{p+1}\\
 %0&=\mathbf{x}_1-\mathbf{x}_2+\mathbf{c}\Delta{t}^{p+1}(1-(1/2)^{p} )\\
  0&=\mathbf{x}_1-\mathbf{x}_2+\mathbf{c}\Delta{t}^{p+1}(1-2^{-p} )\\
 \mathbf{c}&=\frac{\mathbf{x}_1-\mathbf{x}_2}{\Delta{t}^{p+1}(2^{-p} -1)}
\end{flalign*}
Substitute $\mathbf{c}$ into the second estimate to get:
\begin{flalign*}
 \mathbf{x}(t+\Delta{t}) &= \mathbf{x}_1+\frac{\mathbf{x}_1-\mathbf{x}_2}{\Delta{t}^{p+1}(2^{-p} -1)}\Delta{t}^{p+1}
 (1/2)^{p}+\mathcal{O}(\Delta{t}^{p+2})\\
  &= \mathbf{x}_2+\frac{\mathbf{x}_1-\mathbf{x}_2}{2^{p}-1}+\mathcal{O}(\Delta{t}^{p+2})
\end{flalign*}
So if we choose a 4th order method, $p=4$, and to 6th order the truncation error is given 
\begin{equation*}
 \varepsilon=\frac{|\mathbf{x}_1-\mathbf{x}_2|}{2^{p}-1}=\frac{|\mathbf{x}_1-\mathbf{x}_2|}{15}
\end{equation*}
Therefore if we want the truncation error of a given step to be below some value, $\varepsilon_0$:
\begin{flalign*}
 \mathbf{c}\Delta{t}^{p+1}_\text{new}&=\frac{\mathbf{x}_1-\mathbf{x}_2}{(2^{-p} -1)}
 \frac{\Delta{t}^{p+1}_\text{new}}{\Delta{t}^{p+1}} \\
 &=\varepsilon
 \frac{\Delta{t}^{p+1}_\text{new}}{\Delta{t}^{p+1}} \\
 &\leq
 \varepsilon_0
\end{flalign*}
% Therefore can choose $\Delta{t}^{p+1}_\text{new}$ from $\Delta{t}^{p+1}$ by:
% \begin{equation*}
%  \Delta{t}^{p+1}_\text{new}=\Delta{t}^{p+1}\Big(\frac{
%  \varepsilon_0}{\varepsilon}\Big)^{1/(p+1)}
% \end{equation*}
% And then we can try running again. 

\end{document}
